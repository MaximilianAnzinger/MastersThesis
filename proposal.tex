% Configure documentclass
\documentclass[a4paper,12pt,twoside]{article}

% Add common preamble to the document
\input{common/preamble.tex}
\def\proposal{Proposal for}

%%%%%%%%%%%%%%%%%%%%%%%%%%%%%%%%%%%%%%%%%%%%%%%%%%%%%%%%%%%
% Theses specific packages go here
%%%%%%%%%%%%%%%%%%%%%%%%%%%%%%%%%%%%%%%%%%%%%%%%%%%%%%%%%%%
\usepackage[nolist]{acronym}
\begin{acronym}
        \acro{AL}{Adaptive Learning}
        \acro{CIT}{School of Computation, Information and Technology}
        \acro{LMS}{Learning Management System}
        \acro{SRL}{Self-Regulated Learning}
        \acro{TUM}{Technical University of Munich}
\end{acronym}

%%%%%%%%%%%%%%%%%%%%%%%%%%%%%%%%%%%%%%%%%%%%%%%%%%%%%%%%%%%
% Begin of document
%%%%%%%%%%%%%%%%%%%%%%%%%%%%%%%%%%%%%%%%%%%%%%%%%%%%%%%%%%%
\begin{document}
\setlength{\evensidemargin}{22pt}
\setlength{\oddsidemargin}{22pt}


\hypersetup{pdfborder={0 0 0}, pdfauthor={\author}, pdftitle={\title}}

\lstset{showspaces=false, numbers=left, frame=single, basicstyle=\small}

%------- Title setup -------
\include{common/titlepage}

\selectlanguage{english}
\pagenumbering{arabic}

\fancyhead{}
\pagestyle{fancy}
\fancyhead[LE]{\slshape \leftmark}
\fancyhead[RO]{\slshape \rightmark}
\headheight=15pt

%------- Start of Proposal -------
%\TODO{Before you start with your thesis, have a look at our guides on confluence! \url{https://confluence.ase.in.tum.de/display/EduResStud/How+to+thesis}}
\section{Introduction}
%\missing{ Introduction
%        \begin{itemize}
%                \item Introduce the reader to the general setting
%                \item What is the environment?
%                \item What are the tools in use?
%        \end{itemize}
%}

How students learn and how educators design learning environments have changed with the widespread availability of technology. Many educational
institutions have already integrated online learning into their teaching and are exploring new possibilities. \textit{Artemis} is a learning management system for interactive
learning used at multiple universities in Europe.
By allowing instructors to create interactive exercises, automation of feedback as well as grading, the educational tool aims to reduce the
workload of instructors whilst maintaining a high-quality learning experience for students \cite{Krusche_ArTEMiS_An_Automatic_2018}. Fuchs and Waldhauser have already added support
for learning and teaching analytics \cite{fuchs2021,waldhauser2021}. Aberle's contributions extended the possibilities to assign exercises to competencies and create relations
between such, effectively setting the foundation for the integration of \ac{AL} in Artemis \cite{aberle2022}.

% more students, more online, new possibilites
% first steps towards adaptive learning
% competency relations

\section{Problem}
%\missing{ Problem description
%        \begin{itemize}
%                \item What is/are the problem(s)?
%                \item Identify the actors and use these to describe how the problem negatively influences them.
%                \item Do not present solutions or alternatives yet!
%                \item Present the negative consequences in detail
%        \end{itemize}
%}

Large-scale courses with up to 2000 students make it increasingly difficult for instructors to account for students' individual needs. In such large groups,
instructors inevitably encounter students with vastly different backgrounds, abilities, ages, and experiences \cite{mulryan2010teaching}.
One contributing factor to such a heterogeneous student body is the increasing number of international students \cite{oecd2011}. According to
Woessmann the differences in school systems impact students' achievement significantly \cite{woessmann2016importance}.
Currently, Artemis incorporates little to no features that support instructors' efforts to provide students with individualized material. This lack
of individualization may cause students to be either bored or overwhelmed, situations that might impact the learning outcome and even mental-health
\cite{graciani2020m, kadison2004college}. Although the integration of competency management is a first step towards a solution to this pressing issue,
the current implementation is unintuitive, hence, too unattractive to be used by instructors.

Finally, research indicates the importance of \ac{SRL} \cite{bjork2013self}, a term that describes students' capabilities to set goals, examine learning
strategies, and reflect on the outcome \cite{butler1995feedback}. The addition of dashboards for self-evaluation by Waldhauser has improved the availability
of data that students require to plan and apply \ac{SRL} strategies \cite{waldhauser2021}. However, the platform does not actively promote \ac{SRL}.


\section{Motivation}
%\missing{ Thesis Motivation
%        \begin{itemize}
%                \item Outline why it is important to solve the problem
%                \item Again use the actors to present your solution, but don't be to specific
%                \item Be visionary!
%                \item If applicable, motivate with existing research, previous work
%        \end{itemize}
%}

On-demand availability of learning material on Artemis has given students a great degree of freedom to plan their learning experiences. Nonetheless, research
indicates that students may require assistance to do so effectively \cite{latham1991self}. Moreover, Artemis already collects a variety of data that can be used to analyze
the capabilities of individual students. Therefore, this information can be used to provide an adaptive learning experience. Research has shown that
\ac{AL} can improve learning outcomes significantly \cite{liu2017investigating}.

Furthermore, the implementation of \ac{AL} can incorporate features to allow for and promote \ac{SRL}. Therefore, the platform can support students to learn
effectively and perform better \cite{anthonysamy2020self}.

% reduce extraneous load
% self regulated learning
% adaptive learning

\section{Objective}
%\missing{ Thesis Objective
%        \begin{itemize}
%                \item What are the main goals of your thesis?
%        \end{itemize}
%}

This thesis will focus on the integration of \ac{AL} into Artemis.
We will use theoretical research and a requirements analysis to create a software design based on the existing architecture of Artemis. The proposed
design will be implemented in the \ac{LMS}. Finally, the thesis will document the analysis of the current state, the requirements engineering process,
and the development of the features proposed in this section.

\autoref{fig:AOM} shows the analysis object model. We will extend the existing data model by \textit{LearningPaths} to grant students
an individual learning experience by tracking and evaluating students' skill levels. The system will recommend one \textit{LearningObject}
(exercise or lecture unit), for which the student has mastered all required competencies (\textit{CompetencyRelation}), depending on this evaluation.
The features are explained in detail in the consecutive subsections.

\begin{figure}[h!]
        \centering
        \includegraphics[width=\linewidth]{figures/AOM.jpg}
        \caption{Analysis Object Model for the proposed features and use cases (additions are highlighted).}
        \label{fig:AOM}
\end{figure}

\subsection{Competency Management Improvements}
The previously mentioned lack of user-friendliness is an important issue. Hence, we will explore approaches to improve the user interface of the
competency management, e.g., a node editor interface.

To allow for individual learning paths, instructors must be able to configure competencies in more detail. We will add the
functionality that instructors can flag competencies or specific exercises as optional material. Therefore, ambitious students' learning paths
may incorporate these tasks to maintain the relatively high level of difficulty that these students may expect to keep them motivated.
On the other hand, the learning path system can omit these exercises for students that fall behind and provide alternative resources to allow them
to reiterate and gain more confidence with the required learning material.
It should be noted, that both of the before-mentioned features have to exclude exercises that contribute towards the course score.

Finally, we extend competencies by adding the option for instructors to configure deadlines. Therefore, students can get feedback if they
are on track with the expected progress. Use cases related to competency management can be seen in \autoref{fig:UseCasesInstructor}.

\begin{figure}[h!]
        \centering
        \includegraphics[width=0.9\linewidth]{figures/UseCasesInstructor.jpg}
        \caption{Use Cases related to competency- and learning path management (new use cases are highlighted).}
        \label{fig:UseCasesInstructor}
\end{figure}

\subsection{Learning Path Generation}
Once a student enrolls in a course, the system will create an individual learning path. When accessing the learning path for the first time,
students will encounter a self-assessment questionnaire. Hence, the system can perform initial estimates for students' competencies.
The learning path will keep track of the student's progress and evaluate their skill level based on several indicators, e.g. completion, score,
elapsed time, and number of submissions. Based on this evaluation the system can then recommend suitable lecture units or exercises to fit the
individual's needs.

\subsection{Student View}
Students must be able to view their progress in their learning path. We add an intuitive representation of the already completed and upcoming tasks.
Furthermore, students will be able to review whether their progression is within the instructor's expectations regarding time. Further feedback indicators
representing the student's performance compared to other enrolled students can be beneficial to promote self-regulatory learning.

\autoref{fig:LearningPathStudentViewMockup} depicts the view of a student during the learning process. On the left side of the screen, students can see their
progress represented as a network. When selecting one of the nodes, information about the exercise or lecture unit will be displayed. Hence, students can choose
to continue with this specific learning object. Furthermore, the student will be able to continue with the recommendation made by the system (bottom-right arrow)
or navigate back to their previous task. Hence, students spend less time choosing their next task and can fluently
continue throughout their learning experience.
The learning object will then be displayed in the center of the screen. We will reuse the existing exercise view, hence, students continue their learning experience in
a familiar environment. Finally, the Q\&A-section of the learning object will be positioned at the right.

\begin{figure}[h!]
        \centering
        \includegraphics[width=\linewidth]{LearningPathStudentView.jpg}
        \caption{Mockup of the UI when a student is participating via the learning path interface.}
        \label{fig:LearningPathStudentViewMockup}
\end{figure}

\subsection{Learning Analytics}
The application must incorporate a dashboard for instructors to analyze the progress of the participating students
(see \autoref{fig:UseCasesInstructor}). To review the effectiveness of the learning material, instructors need access
to indicators, such as average completion of competencies, average participation in exercises, and average score of exercises.
Furthermore, instructors will be able to get insight into every student's learning paths to allow for individual feedback.


\section{Schedule}
%\missing{ Thesis Schedule
%        \begin{itemize}
%                \item When will the thesis Start (Always 15th of Month)
%                \item Create a rough plan for your thesis (separate the time in sprints with a length of 2-4 Weeks)
%                \item Each sprint should contain several work items - Again keep it high-level and make to keep your plan realistic
%                \item Make sure the work-items are measurable and deliverable
%                \item No writing related tasks! (e.g. "Write Analysis Chapter")
%        \end{itemize}
%}

\begin{itemize}
        \item \textbf{Sprint I} End of May (Week 20-21)
              \begin{itemize}
                      \item Explore methods and existing frameworks to improve the usability of competencies for instructors.
                      \item Implement competency improvements.
              \end{itemize}

        \item \textbf{Sprint II} Mid of June (Week 22-23)
              \begin{itemize}
                      \item Implement competency due dates.
                      \item Implement a mechanism to allow instructors to set competencies as optional content.
                      \item Implement a mechanism to allow instructors to set exercises or lecture units as optional learning material.
              \end{itemize}

        \item \textbf{Sprint III} End of June (Week 24-25)
              \begin{itemize}
                      \item Create server-sided learning path generation for each student.
                      \item Implement a mechanism to let instructors select the accessibility of learning paths.
              \end{itemize}

        \item \textbf{Sprint IV} Beginning of July (Week 26-27)
              \begin{itemize}
                      \item Implement a self-assessment questionnaire.
                      \item Research a method to evaluate students' performance.
              \end{itemize}

        \item \textbf{Sprint V} Beginning of August (Week 28-29)
              \begin{itemize}
                      \item Implement student view of their individual learning path.
                      \item Implement instructor view to visualize the general and individual progress of students.
              \end{itemize}

        \item \textbf{Sprint VI} Mid of August (Week 30-31)
              \begin{itemize}
                      \item Implement the evaluation of students' performance.
                      \item Implement recommendations for learning material based on students' performance.
              \end{itemize}

        \item \textbf{Sprint VII} Beginning of September (Week 32-34)
              \begin{itemize}
                      \item Extend learning path visualization to integrate recommended learning material.
                      \item Implement student view for performance indicators in learning paths.
              \end{itemize}
\end{itemize}

\clearpage
\begin{acronym}
    \acro{GUI}{Graphical User Interface}

    \acro{AL}{Adaptive Learning}
    \acro{CIT}{School of Computation, Information and Technology}
    \acro{LMS}{Learning Management System}
    \acro{SRL}{self-Regulated Learning}
    \acro{TUM}{Technical University of Munich}
\end{acronym}

\clearpage
\bibliography{thesis}
\bibliographystyle{alpha}

\end{document}
