% Configure documentclass
\documentclass[a4paper,12pt,twoside]{article}

% Add common preamble to the document
%% This file is shared between thesis and proposal

\usepackage[scaled]{helvet}
\usepackage{url}
\usepackage{cite}
\usepackage{listings}
\usepackage[pdftex]{graphicx}
\usepackage[hang,small,bf]{caption}
\usepackage{styles/tum}
\usepackage{setspace}
\usepackage[german,english]{babel}
\usepackage{float}
\usepackage{floatflt}
\usepackage{fancyhdr}
\usepackage{color}
\usepackage{booktabs}
\usepackage[pdftex,bookmarks=true,plainpages=false,pdfpagelabels=true]{hyperref}	%TODO make yourself familiar with \label, \ref and \hyperref for referencing figures, tables, chapters, etc.
\usepackage{mdwlist}
\usepackage{enumerate}
\usepackage{array}
\usepackage{longtable}
\usepackage[utf8]{inputenc}
\usepackage[capitalize, noabbrev]{cleveref}
\usepackage{wasysym}
\usepackage{todonotes}

% Path for graphics
\graphicspath{{figures/}}

% Include the Thesis metadata like title, author, etc. 
\input{metadata}

%%%%%%%%%%%%%%%%%%%%%%%%%%%%%%%%%%%%%%%%%%%%%%%%%%%%%%%%%%%%%%%%%%%%%%%%%%%%%%%%%%%%%%%%%%%%%%%%%
% Custom Commands for this template
%%%%%%%%%%%%%%%%%%%%%%%%%%%%%%%%%%%%%%%%%%%%%%%%%%%%%%%%%%%%%%%%%%%%%%%%%%%%%%%%%%%%%%%%%%%%%%%%%

% Annotate feedback you received 
\newcommand{\feedback}[1]{\todo[inline,color=green,caption={}]{#1}}

% State what is missing in this spot
\newcommand{\missing}[1]{\todo[inline,color=yellow,caption={}]{#1}}

% Inline to do note: 
\newcommand{\TODO}[1]{\todo[inline,caption={}]{#1}}





\def\proposal{Proposal for}

%%%%%%%%%%%%%%%%%%%%%%%%%%%%%%%%%%%%%%%%%%%%%%%%%%%%%%%%%%%
% Theses specific packages go here
%%%%%%%%%%%%%%%%%%%%%%%%%%%%%%%%%%%%%%%%%%%%%%%%%%%%%%%%%%%
\usepackage[nolist]{acronym}

%%%%%%%%%%%%%%%%%%%%%%%%%%%%%%%%%%%%%%%%%%%%%%%%%%%%%%%%%%%
% Begin of document
%%%%%%%%%%%%%%%%%%%%%%%%%%%%%%%%%%%%%%%%%%%%%%%%%%%%%%%%%%%
\begin{document}
\setlength{\evensidemargin}{22pt}
\setlength{\oddsidemargin}{22pt}


\hypersetup{pdfborder={0 0 0}, pdfauthor={\author}, pdftitle={\title}}

\lstset{showspaces=false, numbers=left, frame=single, basicstyle=\small}

%------- Title setup -------
\thispagestyle{empty}
{
\sffamily

\vspace{1cm}
\begin{center}
\oTUM{4cm}

\vspace{5mm}     
{\LARGE \bf \sffamily Technical University of Munich}

\vspace{5mm}
{\Large School of Computation, Information and Technology \\ -- Informatics -- }	
\vspace{1mm}
\end{center}

\vspace{15mm}

\begin{center}
        {\large {\proposal} {\degree}'s Thesis in \program}
\vspace{8mm}

\begin{spacing}{1.3}
{\LARGE \bf \sffamily \title}\\
\vspace{8mm}

{\LARGE \titleGer}\\
\vspace{8mm}
\end{spacing}

\begin{tabular}{ll}
\large Author:           & \large \author     \\[2mm]
\large Supervisor:       & \large \supervisor \\[2mm]				
\large Advisor:	         & \large \advisor    \\[2mm]
\ifx\proposal\empty\else
\large Start Date:       & \large \startdate  \\[2mm]
\fi
\large Submission Date:  & \large \date
\end{tabular}

\end{center}
}


\selectlanguage{english}
\pagenumbering{arabic}

\fancyhead{}
\pagestyle{fancy}
\fancyhead[LE]{\slshape \leftmark}
\fancyhead[RO]{\slshape \rightmark}
\headheight=15pt

%------- Start of Proposal -------
\TODO{Before you start with your thesis, have a look at our guides on confluence! \url{https://confluence.ase.in.tum.de/display/EduResStud/How+to+thesis}}
\section{Introduction}
\missing{ Introduction
        \begin{itemize}
                \item Introduce the reader to the general setting
                \item What is the environment?
                \item What are the tools in use?
        \end{itemize}
}

The wide availability of technology has changed the way students learn as well as the design of learning environments by instructors.

% more students, more online, new possibilites
% first steps towards adaptive learning
% competency relations

\section{Problem}
\missing{ Problem description
        \begin{itemize}
                \item What is/are the problem(s)?
                \item Identify the actors and use these to describe how the problem negatively influences them.
                \item Do not present solutions or alternatives yet!
                \item Present the negative consequences in detail
        \end{itemize}
}



\section{Motivation}
\missing{ Thesis Motivation
        \begin{itemize}
                \item Outline why it is important to solve the problem
                \item Again use the actors to present your solution, but don't be to specific
                \item Be visionary!
                \item If applicable, motivate with existing research, previous work
        \end{itemize}
}

% reduce extraneous load

\section{Objective}
\missing{ Thesis Objective
        \begin{itemize}
                \item What are the main goals of your thesis?
        \end{itemize}
}

This thesis will

\subsection{Competency Management Improvements}
% flag competencies & exercises/units as optional
% bundel exercises (different difficulties)
To allow for individual learning paths, instructors must be able to configure competencies in more detail. Firstly, instructors should be
able to bundle exercises that represent a variety of different difficulty levels, s.t. the learning path generation can lateron cater exercises
with an appropriate complexity to the individual students.

Furthermore, instructors should be able to flag compentencies or specific exercises as optional material. Therefore, ambitious students' learning paths
may incorporate these tasks to maintain the relatively high level of difficulty that these students may expect to keep them motivated.
On the other hand, the learning path system can ommit these exercises for students that fall behind and provide alternative resources to allow them
to reiterate and gain more confidence with the required learning material.
It should be noted, that both of the before mentioned features should exclude exercises that contribute towards the course score.

Finally, insturctors should be able to configure deadlines for individual competencies. Therefore, students can get feedback if they are on track with the
expected progress.

\subsection{Learning Path Generation}
Once a student enrolls in a course, the system should create an individual learning path. The

The system should provide reccomendations,

\subsection{Student View}
Students must be able to view their progress in their individual learning path. There should be an intuitive representation of the already completed and upcoming tasks.
Furthermore, students should be able to review wheather their progression is within the instructors expections in regards of time. Further feedback indicators
representing the students performance compared to other enrolled students might be beneficial to promote self-regulatory learning.
The overview should also incorporate the reccomendation made by the learning path generation. Hence, students spend less time choosing their next task and can fluently
continue througout their learning experience.

\subsection{Learning Analytics}
The application must incorporate a dashboard for instructors to analyze the progress of the participating students. To review the effectiveness of the learning material,
instructors need access to indicators, such as average completion of competencies, average participation in exercises, and average score of exercises.
Furthermore, insturctors should be able to get insight into individual students' learning path to allow for individual feedback.


\section{Schedule}
\missing{ Thesis Schedule
        \begin{itemize}
                \item When will the thesis Start (Always 15th of Month)
                \item Create a rough plan for your thesis (separate the time in sprints with a length of 2-4 Weeks)
                \item Each sprint should contain several work items - Again keep it high-level and make to keep your plan realistic
                \item Make sure the work-items are measurable and deliverable
                \item No writing related tasks! (e.g. "Write Analysis Chapter")
        \end{itemize}
}

\begin{itemize}
        \item \textbf{Sprint I} End of May (Week 20-21): Improve Reusability
              \begin{itemize}
                      \item Instructors can select wheather their material can be reused by other instructors
                      \item Instructors can search for material using competencies
              \end{itemize}

        \item \textbf{Sprint II} Mid of June (Week 22-23): asdf
              \begin{itemize}
                      \item asdf
              \end{itemize}

        \item \textbf{Sprint } asdf (asdf): Improve Competency Management
              \begin{itemize}
                      \item Instructors can
              \end{itemize}

        \item \textbf{Sprint } asdf (asdf): Create Student View
              \begin{itemize}
                      \item asdf
              \end{itemize}

        \item \textbf{Sprint } asdf (asdf): Basic Adaptive Learning
              \begin{itemize}
                      \item Students get reccomendations for learning material (e.g., exercises, lecture units) based on their current skill level
              \end{itemize}
\end{itemize}

\clearpage
\begin{acronym}
    \acro{GUI}{Graphical User Interface}

    \acro{AL}{Adaptive Learning}
    \acro{CIT}{School of Computation, Information and Technology}
    \acro{LMS}{Learning Management System}
    \acro{SRL}{self-Regulated Learning}
    \acro{TUM}{Technical University of Munich}
\end{acronym}

\clearpage
\bibliography{thesis}
\bibliographystyle{alpha}

\end{document}
